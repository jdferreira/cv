\section{Pedagogical performance}

\subsection{Teaching}

\subsubsection{Courses}
\vspace{-1.5\baselineskip}
\begin{table}[H]
\centering
\caption{This table shows my contribution to courses from the Computer Science Department in Faculty of Sciences, University of Lisbon.\hfill\break
\textbf{Legend}:
\textit{PL}: Laboratory classes;
\textit{TP}: Practical lectures;
\textit{T}:  Theoretical lectures}
\label{tab:courses}
\begin{tabular}{llll}
\toprule
\bfseries Course & \bfseries 2015--2016 & \bfseries 2016--2017 & \bfseries 2017--2018 \\
\midrule
IC   & TP \& PL & --       & PL \& TP \\
IPM  & --       & --       & TP \\
ADSI & --       & --       & TP \\
ITW  & TP \& PL & TP \& PL & TP \& PL \\
ASW  & PL       & PL       & -- \\
PD   & TP       & TP       & TP \\
Prog & TP       & TP       & -- \\
\bottomrule
\end{tabular}
\end{table}

\entry
    [2015--2018]
    {IC (Computer Interaction)}
\begin{itemize}
    \item \textbf{Degree}: B.Sc. in Information Technologies (2\textsuperscript{nd} year)
    \item \textbf{My classes}: TP \& PL
    \item \textbf{Topics}:
    \begin{itemize}
        \item Introduction to Human-Computer Interaction (HCI);
        \item The foundations of HCI: Human and technological aspects;
        \item The design process: user centred design, interaction design basics, guidelines for interaction design, and evaluation techniques;
        \item Models and theories: cognitive models, task analysis, dialogue notations
    \end{itemize}
    \item \textbf{My contributions}: Laboratory materials and project descriptions.
\end{itemize}

\entry
    [2017--2018]
    {IPM (Human-Computer Interfaces)}
\begin{itemize}
    \item \textbf{Degree}: Informatics Engineering (2\textsuperscript{nd} year)
    \item \textbf{My classes}: TP
    \item \textbf{Topics}:
    \begin{itemize}
        \item The fundamental concepts of communication between humans and computers;
        \item Technologies and styles of interaction;
        \item Methods, principles and techniques for task analysis and design of interactive systems;
        \item Usability evaluation;
        \item The iterative cycle.
    \end{itemize}
    \item \textbf{My contributions}: Laboratory materials and project descriptions.
\end{itemize}

\entry
    [2017--2018]
    {ADSI (Information Systems Analysis and Design)}
\begin{itemize}
    \item \textbf{Degree}: Informatics Engineering (3\textsuperscript{rd} year)
    \item \textbf{My classes}: TP
    \item \textbf{Topics}:
    \begin{itemize}
        \item Data and functional analysis;
        \item Analysis of human, organizational and environmental contexts in the context of information systems design;
        \item ``Understanding the problem'' -- emphasizing contact with the sources and analyzing a problem's different perspectives;
        \item ``Developing a solution'' -- searching for an innovative, useful and well projected solution to the identified problem.
    \end{itemize}
\end{itemize}

\entry
    [2015--2017]
    {ITW (Introduction to Web Technologies)}
\begin{itemize}
    \item \textbf{Degree}: B.Sc. in Information Technologies (1\textsuperscript{st} year)
    \item \textbf{My classes}: TP \& PL
    \item \textbf{Topics}:
    \begin{itemize}
        \item The fundamental characteristics of the Web and associated technologies
        \item The models and architectures that support the Web.
        \item The main protocols (e.g. HTTP), specification and web programming languages (HTML, CSS, javaScript, etc.)
        \item Current platforms (e.g. W3.CSS, jQuery) that shape the Web.
    \end{itemize}
    \item \textbf{My contributions}: Class materials used to teach HTML, CSS and JavaScript; design of project ideas based on the development of games as websites.
\end{itemize}

\entry
    [2015--2017]
    {ASW (Web Applications and Services)}
\begin{itemize}
    \item \textbf{Degree}: B.Sc. in Information Technologies (2\textsuperscript{nd} year)
    \item \textbf{My classes}: Laboratory classes
    \item \textbf{Topics}:
    \begin{itemize}
        \item Web application characteristics and features;
        \item The development process of web applications;
        \item Introduction to the main server-side web technologies: resource addressing, protocols and general architecture;
        \item The various data transfer formats (XML, JSON, \emph{etc.}) and related technologies;
        \item Introduction to Web Services and Semantic Web.
    \end{itemize}
    \item \textbf{My contributions}: Tutorial guides teaching how to deal with HTML forms and user input with Javascript.
\end{itemize}

\entry
    [2015--2018]
    {PD (Data Processing)}
\begin{itemize}
    \item \textbf{Degree}: B.Sc. in Biology (2\textsuperscript{nd} year)
    \item \textbf{My classes}: TP
    \item \textbf{Topics}:
    \begin{itemize}
        \item Introduction to data processing;
        \item Introduction to Python (data types and data structures);
        \item Introduction to Regular Expressions;
        \item Introduction to Biomedical Web Services;
        \item Database management systems.
    \end{itemize}
    \item \textbf{My contributions}: I created all the practical lecture materials -- a tutorial to guide students in processing large datasets, with emphasis on a collection of protein and metabolic information extracted from widely-known biomedical web services, including protein sequences and metabolic pathway data. The goal was to teach students how to process biology-related data with a programming language and an underlying database. Git repository in \url{https://github.com/jdferreira/data-processing-book}.
\end{itemize}


\entry
    [2015--2017]
    {Prog (Programming 1)}
\begin{itemize}
    \item \textbf{Degree}: Several B.Sc and M.Sc courses offered at Faculty of Sciences, University of Lisbon
    \item \textbf{My classes}: TP
    \item \textbf{Topics}:
    \begin{itemize}
        \item Computation: computability and Turing machines;
        \item Algorithms: exhaustive search, approximation search and bisection search;
        \item Programming methods: attribution and verification, decision, iteration and recursion, abstraction and specification, cloning;
        \item Programming languages: expressions and types, precedence and associativity, functions, scope, libraries and modules;
        \item Data structures: sequences, tuples, lists and dictionaries;
        \item Files;
        \item Software development: reading and writing, documentation, assertions and exceptions, test and debugging.
    \end{itemize}
    \item \textbf{My contributions}: I contributed to the practice class materials by suggesting new exercises and modifications to existing ones.
\end{itemize}


\subsubsection{Pedagogical surveys}

Lecturers, in Faculty of Sciences, University of Lisbon, are graded at the end of the semester by their students. These are my results. Results from the current year are not available yet.

\begin{table}[H]
\centering
\caption{Pedagogical survey results. Students could either not answer each question or answer from 1 (strong disagreement) to 4 (strong agreement). Results for each question show the average for the students that answered the question. Results from the academic year of 2016--2017 are still unavailable.\hfill\break
\textbf{Legend}:\hfill\break
\textit{Q1}:~Did the professor lecture with clarity?\hfill\break
\textit{Q2}:~Did the professor answer questions with clarity?\hfill\break
\textit{Q3}:~Was the professor available for outside-of-class contact \& support?\hfill\break
\textit{Q4}:~Was there a good pedagogical relation between professor and students?\hfill\break
\textit{Q5}:~What is your global appreciation of the professor?\hfill\break
\textit{PL}:~Laboratory class;\hfill\break
\textit{TP}:~Practical lecture;\hfill\break
\textit{TP}:~Theoretical lecture;\hfill\break
\textit{$*$}:~Course still ongoing, results unavailable.}
\label{tab:surveys}
\vspace{5mm}
\begin{tabular}{lcccccc}
\toprule
\bfseries 2015--2016
 & \multicolumn{2}{c}{\bfseries IC}
 & \bfseries Prog I
 & \bfseries ASW
 & \multicolumn{2}{c}{\bfseries ITW} \\
\bfseries Question
 & \bfseries TP
 & \bfseries PL
 & \bfseries TP
 & \bfseries PL
 & \bfseries TP
 & \bfseries PL \\
\midrule
Q1 & 3.86 & 3.86 & 3.69 & 3.76 & 3.68 & 3.70 \\
Q2 & 3.86 & 3.80 & 3.72 & 3.84 & 3.58 & 3.70 \\
Q3 & 3.70 & 3.71 & 3.62 & 3.87 & 3.69 & 3.94 \\
Q4 & 3.93 & 3.86 & 3.80 & 3.88 & 3.63 & 3.85 \\
Q5 & 3.87 & 3.88 & 3.79 & 3.84 & 3.70 & 3.73 \\
\bottomrule
\end{tabular}

\vspace{5mm}

\begin{tabular}{lccccc}
\toprule
\bfseries 2016--2017
 & \bfseries PD
 & \bfseries Prog I
 & \bfseries ASW
 & \multicolumn{2}{c}{\bfseries ITW} \\
\bfseries Question
 & \bfseries TP
 & \bfseries TP
 & \bfseries PL
 & \bfseries TP
 & \bfseries PL \\
\midrule
Q1 & 3.42 & 3.53 & 3.53 & 3.72 & 3.57 \\
Q2 & 3.50 & 3.61 & 3.59 & 3.71 & 3.75 \\
Q3 & 3.39 & 3.57 & 3.36 & 3.74 & 3.66 \\
Q4 & 3.45 & 3.55 & 3.50 & 3.74 & 3.71 \\
Q5 & 3.44 & 3.55 & 3.55 & 3.79 & 3.75 \\
\bottomrule
\end{tabular}
\end{table}


\subsection{Jury and Examinations}

\entry
    [2018]
    {M.Sc in ``Informatics Engineering''}
\begin{itemize}
    \item \textbf{Student}: Pedro Davim Teixeira Mendes
    \item \textbf{Title}: ``Development of custom solutions in Sharepoint'' (loosely translated from the Portuguese official title ``Desenvolvimento de soluções à medida em Sharepoint'')
    \item \textbf{Supervisor}: Dr Luis Antunes (Faculty of Sciences, University of Lisbon)
\end{itemize}

\entry
    [2017]
    {M.Sc in ``Informatics Engineering''}
\begin{itemize}
    \item \textbf{Student}: Tiago Alexandre Fernandes de Noronha
    \item \textbf{Title}: ``Mobile application development and support services'' (loosely translated from the Portuguese official title ``Desenvolvimento de aplicação móvel e serviços de suporte'')
    \item \textbf{Supervisor}: Dr João Balsa Silva (Faculty of Sciences, University of Lisbon)
\end{itemize}

\entry
    [2017]
    {M.Sc in ``Information Systems and Computer Engineering''}
\begin{itemize}
    \item \textbf{Student}: Sebastião da Silva Freire
    \item \textbf{Title}: ``E-Sports Ontology''
    \item \textbf{Supervisor}: Dr H Sofia Pinto (IST, University of Lisbon)
    \item I was formally invited by the supervisor; there has already been a discussion on the proposal of the M.Sc, which I examined in February 2017.
\end{itemize}

\entry
    [2017]
    {M.Sc in ``Mathematics Applied to Economics and Business''}
\begin{itemize}
    \item \textbf{Student}: Catarina Nunes Valente
    \item \textbf{Title}: ``Excel programming for Statistics: Linear Models and Extensions'' (loosely translated from the Portuguese official ``Programação em Excel para Estatística: Modelo Linear e Extensões'')
    \item \textbf{Supervisor}: Dr Teresa Alpuim (Faculty of Sciences, University of Lisbon)
\end{itemize}

\entry
    [2016]
    {M.Sc in ``Bioinformatics and Computational Biology''}
\begin{itemize}
    \item \textbf{Student}: Samuel Viana
    \item \textbf{Title}: ``Optimizing 16S Sequencing Analysis Pipelines''
    \item \textbf{Supervisors}: Daniel Faria (Instituto Gulbenkian de Ciência) and Catia Pesquita (Faculty of Sciences, University of Lisbon)
\end{itemize}


\subsection{Teaching-related activities}

\subsubsection{Invited participation in courses}

\entry
    [2015-2017]
    {Big Data}
\begin{itemize}
    \item \textbf{Degree}: Ph.D. in Informatics
    \item I supervised the lectures and journal club for the course regarding the topics of machine learning in Big Data and the semantic web in Big Data.
\end{itemize}

\entry
    [2014--2015]
    {AW (Web Applications)}
\begin{itemize}
    \item \textbf{Degree}: M.Sc. in Informatics Engineering
    \item I presented lectures on Semantic Web (SW) with the following topics:
    \begin{itemize}
        \item The problem of ambiguity that SW tries to solve;
        \item Rule-based inference;
        \item RDF statements;
        \item Several of the SW languages (RDF, OWL, SPARQL);
        \item Objects \emph{vs.} Classes \emph{vs.} Instances
        \item An introduction to several of the layers of the SW, including URIs, XML, RDF, Ontologies and Rules;
        \item Real-world examples of SW in action: Semantic wikis, FOAF project, RDFa, hCalendar, Linked Data Project.
    \end{itemize}
\end{itemize}

\entry
    [2013--2014]
    {Bioinformatics \& Computational Modelling}
\begin{itemize}
    \item \textbf{Degree}: Ph.D. Program in Biological Systems -- Functional \& Integrative Genomics
    \item I presented a practical lecture on Bioinformatics, specifically on the use of Python in biomedical data processing. This included:
    \begin{itemize}
        \item Basic python datatypes and functions;
        \item Introduction to BioPython, a package with access to several functions dedicated (i)~to biological data processing and (ii)~to widely-known biomedical web services;
        \item Exercises directed at learning the inners of BioPython, specifically to process protein sequences (using web services to access the SwissProt database and the BLAST algorithm)
        \item Introduction to the Gene Ontology and Semantic Similarity
    \end{itemize}
\end{itemize}

\entry
    [2013--2014]
    {Biomedical Ontologies}
\begin{itemize}
    \item \textbf{Degree}: M.Sc. class available to several M.Sc. students at Faculty of Sciences, University of Lisbon
    \item I collaborated on the practical classes by designing a project and supervising a group of students in implementing the project's specifications. The projects were designed so that students would develop an intuition about ontology development, ontology matching, and semantic similarity. Two of these projects resulted in publications at national and international level.
\end{itemize}

\entry
    [2011--2012]
    {Bioinformatics}
\begin{itemize}
    \item \textbf{Degree}: B.Sc. class available to several B.Sc. students at Faculty of Sciences, University of Lisbon
    \item I was a teaching assistant on the practical classes by invitation of the course's head professor. I created a project specification based on ontology development and text-mining, and supervised the students in their implementations.
\end{itemize}


